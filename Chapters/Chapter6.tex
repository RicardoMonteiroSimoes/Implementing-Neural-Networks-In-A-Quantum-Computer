% Chapter 6

\chapter{Introduction} % Main chapter title

\label{chapter:introduction}

Can a quantum classifier beat the classical variant? This question can be asked for multiple problems and summarized as \emph{"Can quantum $x$ beat the classical variant?"}. For a long time, all possible gains remained in the theoretical realm\cite{shor_polynomial-time_1997}, but current advances in the field of quantum computing have lead to solutions and proposals\cite{farhi_quantum_2014, fankhauser_multiple_2021, havlicek_supervised_2019} that can be programmed and used \emph{today}. At the same time quantum computing is getting more and more available to non-researchers as IBM, Microsoft, Google and co. fight for dominance in this new field of computation. \par
The training of neural networks demands an ever increasing amount of computational power\cite{openai_ai_2018} as complexity of circuits increase. This complexity increase is dictated by the ever more complex problems they are trying to solve, which implies that advancements stand still until further computational power is available.  The goal of this paper is to evaluate current quantum offerings to solve classification problems. In a first step, the application of features and weights onto a quantum gates and their behaviour is assessed. Subsequently,  To assess the potential of such an implementation, two possible solutions are evaluated. One solution replicates the basic arithmetic operations computational neurons use, and one uses quantum specific super positions and entanglement. Due to the necessary amount of qubits for the arithmetic solution, as well as the absence of quantum specific operations, only the the classifier based on entanglement and superpositions is further developed.\par 
Using a variety of different quantum circuits, as well as three different datasets, measurements are done to determine the viability of these designs as well as to compare them to a basic \code{MLP} classifier. The results show that the right combination of quantum gates leads to a more flexible classifier, that can adapt and solve different problem spaces. One problem that accompanies these design is the disparity when it comes to achieved accuracy. Under reproducible circumstances, no two training results are equal. This problem comes directly from the vast expressability such a circuit offers \- by making the problem space much wider, it gets harder to traverse and therefor, find a viable solution. The findings show that quantum circuits can offer a substantial gain when compared to the classical solutions, but still need more refinement before being touted as a viable alternative, without regards to hardware availability.