% Chapter 6

\chapter{Introduction} % Main chapter title

\label{chapter:introduction}

For a long time, the ever occurring question \say{Can quantum $x$ beat the classical variant?} was always met with the answer \say{theoretically, yes}. During this time, all possible gains remained in the theoretical realm\cite{shor_polynomial-time_1997}, but recent advances in the field of quantum computing have brought out solutions and proposals\cite{farhi_quantum_2014, fankhauser_multiple_2021, havlicek_supervised_2019} that can be programmed and used \emph{today}. At the same time, quantum computing is getting more and more available to non-researchers as IBM, Microsoft, Google and co. fight for dominance in this new segment of computation.\par
The training of neural networks to solve the hardest challenges demands an ever-increasing amount of computational power\cite{openai_ai_2018}. The goal of this paper is to evaluate current quantum offerings to solve classification problems, as well as analyse their performance. In a first step, a broad overview of quantum computing is created, as well as the application of features and weights onto a quantum gates assessed. Subsequently, two possible designs are evaluated, the first being a solution that replicates the basic arithmetic operations computational neurons use, and the second one which uses quantum specific superpositions and entanglement. Due to the necessary amount of qubits for the arithmetic solution, as well as the absence of quantum specific operations which make quantum supremacy possible, only the second solution further developed and evaluated.\par 
Using a variety of different quantum circuits, as well as three different datasets, training is done on the simulator to determine the viability of these designs as well as to compare them to a basic \code{MLP} classifier. The results show that the right combination of quantum gates leads to multiple classifiers that can be optimized for different problem spaces. One issue that accompanies these design is the discrepancy when it comes to achieved accuracy. Under reproducible circumstances, no two training results are equal. This issue comes directly from the vast expressibility such a circuit offers - by making the problem space much wider, it gets harder to traverse and therefore, find a viable solution.\par 
When running the trained circuit on real hardware for a given dataset, another issue arises as the noise affecting the real devices disrupts the operations and measurements, which in turn leads to vastly inferior accuracy. Nevertheless, our results suggest that classification with quantum circuits is not only possible, but also viable as a general alternative.