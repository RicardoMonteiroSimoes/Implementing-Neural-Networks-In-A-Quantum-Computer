% Chapter 4

\chapter{Results} % Main chapter title

\label{chapter:results}

The results in this chapter are in condensed form. Please refer to appendix \ref{appendix:boxplots} for more detailed figures. The following figures display the achieved accuracy on the $y$ axis, and the number of the circuit on the $x$ axis, and are limited to only the achieved testing accuracy. For the randomized segments, 10 training runs had its own data shuffling, which are equal for every circuit. Additionally, the IRIS dataset is also classified on a basic MLP classifier.
%----------------------------------------------------------------------------------------
\def \newboxplotwidth {215pt}


\clearpage
\section{Non Randomized IRIS}
\label{chapter:iris_non_randomized}

\begin{figure}[!h]
    \begin{subfigure}{.5\textwidth}
        \centering
        \includesvg[width = \newboxplotwidth]
        {Appendices/boxplots/iris/non_randomized_split/Comparison_of_all_circuits_with_non_randomized,_raw_data.svg}
    \end{subfigure}
    \begin{subfigure}{.5\textwidth}
        \centering
        \includesvg[width = \newboxplotwidth]
        {Appendices/boxplots/iris/non_randomized_split/Comparison_of_all_circuits_with_non_randomized,_normalized_data.svg}
    \end{subfigure}
    \begin{subfigure}{.5\textwidth}
        \centering
        \includesvg[width = \newboxplotwidth]
        {Appendices/boxplots/iris/non_randomized_split/Comparison_of_all_circuits_with_non_randomized,_pified_data.svg}
    \end{subfigure}
    \begin{subfigure}{.5\textwidth}
        \centering
        \includesvg[width = \newboxplotwidth]
        {Appendices/boxplots/iris/non_randomized_split/MLP_on_non_randomized,_raw_data.svg}
    \end{subfigure}
    \begin{subfigure}{.5\textwidth}
        \centering
        \includesvg[width = \newboxplotwidth]
        {Appendices/boxplots/iris/non_randomized_split/MLP_on_non_randomized,_normalized_data.svg}
    \end{subfigure}
    \begin{subfigure}{.5\textwidth}
        \centering
        \includesvg[width = \newboxplotwidth]
        {Appendices/boxplots/iris/non_randomized_split/MLP_on_non_randomized,_pified_data.svg}
    \end{subfigure}
    \caption{Comparison of test results from all trained circuits on the non-randomized IRIS dataset}
    \label{fig:circuits_results_non_r_iris}
\end{figure}
\clearpage

\section{Randomized IRIS}
\label{chapter:iris_randomized}

\begin{figure}[!h]
    \begin{subfigure}{.5\textwidth}
        \centering
        \includesvg[width = \newboxplotwidth]
        {Appendices/boxplots/iris/randomized_split/Comparison_of_all_circuits_with_randomized,_raw_data.svg}
    \end{subfigure}
    \begin{subfigure}{.5\textwidth}
        \centering
        \includesvg[width = \newboxplotwidth]
        {Appendices/boxplots/iris/randomized_split/Comparison_of_all_circuits_with_randomized,_normalized_data.svg}
    \end{subfigure}
    \begin{subfigure}{.5\textwidth}
        \centering
        \includesvg[width = \newboxplotwidth]
        {Appendices/boxplots/iris/randomized_split/Comparison_of_all_circuits_with_randomized,_pified_data.svg}
    \end{subfigure}
    \begin{subfigure}{.5\textwidth}
        \centering
        \includesvg[width = \newboxplotwidth]
        {Appendices/boxplots/iris/randomized_split/MLP_on_randomized,_raw_data.svg}
    \end{subfigure}
    \begin{subfigure}{.5\textwidth}
        \centering
        \includesvg[width = \newboxplotwidth]
        {Appendices/boxplots/iris/randomized_split/MLP_on_randomized,_normalized_data.svg}
    \end{subfigure}
    \begin{subfigure}{.5\textwidth}
        \centering
        \includesvg[width = \newboxplotwidth]
        {Appendices/boxplots/iris/randomized_split/MLP_on_randomized,_pified_data.svg}
    \end{subfigure}
    \caption{Comparison of test results from all trained circuits on the randomized IRIS dataset}
    \label{fig:circuits_results_r_iris}
\end{figure}

\clearpage
\subsection{Classification Of IRIS On Quantum Hardware}

\begin{figure}[!h]
    \centering
    \includesvg[width = \newboxplotwidth]{Appendices/boxplots/real_hardware/Accuracy_of_trained_circuit_1_on_real_quantum_hardware.svg}
    \caption{Achieved accuracy of circuit 1 on real quantum hardware, on the randomized IRIS dataset}
    \label{fig:boxplot_real_hardware}
\end{figure}

\clearpage 
\section{Non Randomized Heart Failure Prediction}
\label{chapter:heart_failure_prediction_non_randomized}

\begin{figure}[!h]
    \begin{subfigure}{.5\textwidth}
        \centering
        \includesvg[width = \newboxplotwidth]
        {Appendices/boxplots/heart_failure/non_randomized_split/Comparison_of_all_circuits_with_non_randomized,_raw_data.svg}
    \end{subfigure}
    \begin{subfigure}{.5\textwidth}
        \centering
        \includesvg[width = \newboxplotwidth]
        {Appendices/boxplots/heart_failure/non_randomized_split/Comparison_of_all_circuits_with_non_randomized,_normalized_data.svg}
    \end{subfigure}
    \begin{subfigure}{.5\textwidth}
        \centering
        \includesvg[width = \newboxplotwidth]
        {Appendices/boxplots/heart_failure/non_randomized_split/Comparison_of_all_circuits_with_non_randomized,_pified_data.svg}
    \end{subfigure}
    \caption{Comparison of test results from all trained circuits on the non-randomized Heart Failure dataset}
    \label{fig:circuits_results_non_r_hf}
\end{figure}

\clearpage 
\section{Randomized Heart Failure Prediction}
\label{chapter:heart_failure_prediction_randomized}

\begin{figure}[!h]
    \begin{subfigure}{.5\textwidth}
        \centering
        \includesvg[width = \newboxplotwidth]
        {Appendices/boxplots/heart_failure/randomized_split/Comparison_of_all_circuits_with_randomized,_raw_data.svg}
    \end{subfigure}
    \begin{subfigure}{.5\textwidth}
        \centering
        \includesvg[width = \newboxplotwidth]
        {Appendices/boxplots/heart_failure/randomized_split/Comparison_of_all_circuits_with_randomized,_normalized_data.svg}
    \end{subfigure}
    \begin{subfigure}{.5\textwidth}
        \centering
        \includesvg[width = \newboxplotwidth]
        {Appendices/boxplots/heart_failure/randomized_split/Comparison_of_all_circuits_with_randomized,_pified_data.svg}
    \end{subfigure}
    \caption{Comparison of test results from all trained circuits on the randomized Hearth Failure dataset}
    \label{fig:circuits_results_r_hf}
\end{figure}

\clearpage 
\section{Non-randomized Artificial Problem}
\label{chapter:artificial_problem_non_randomized}

\begin{figure}[!h]
    \begin{subfigure}{.5\textwidth}
        \centering
        \includesvg[width = \newboxplotwidth]
        {Appendices/boxplots/artificial_problem/non_randomized_split/Comparison_of_all_circuits_with_non_randomized,_raw_data.svg}
    \end{subfigure}
    \begin{subfigure}{.5\textwidth}
        \centering
        \includesvg[width = \newboxplotwidth]
        {Appendices/boxplots/artificial_problem/non_randomized_split/Comparison_of_all_circuits_with_non_randomized,_normalized_data.svg}
    \end{subfigure}
    \begin{subfigure}{.5\textwidth}
        \centering
        \includesvg[width = \newboxplotwidth]
        {Appendices/boxplots/artificial_problem/non_randomized_split/Comparison_of_all_circuits_with_non_randomized,_pified_data.svg}
    \end{subfigure}
    \caption{Comparison of test results from all trained circuits on the non-randomized Artificial Problem dataset}
    \label{fig:circuits_results_non_r_ap}
\end{figure}

\clearpage 
\section{Randomized Artificial Problem}
\label{chapter:artificial_problem_randomized}
\begin{figure}[!h]
    \begin{subfigure}{.5\textwidth}
        \centering
        \includesvg[width = \newboxplotwidth]
        {Appendices/boxplots/artificial_problem/randomized_split/Comparison_of_all_circuits_with_randomized,_raw_data.svg}
    \end{subfigure}
    \begin{subfigure}{.5\textwidth}
        \centering
        \includesvg[width = \newboxplotwidth]
        {Appendices/boxplots/artificial_problem/randomized_split/Comparison_of_all_circuits_with_randomized,_normalized_data.svg}
    \end{subfigure}
    \begin{subfigure}{.5\textwidth}
        \centering
        \includesvg[width = \newboxplotwidth]
        {Appendices/boxplots/artificial_problem/randomized_split/Comparison_of_all_circuits_with_randomized,_pified_data.svg}
    \end{subfigure}
    \caption{Comparison of test results from all trained circuits on the randomized Artificial Problem dataset}
    \label{fig:circuits_results_r_ap}
\end{figure}

\clearpage
\section{Discussion}
The results when training is done with static training/test data shuffle show that due to the nature of probabilistic measurements, one cannot \emph{naturally} expect to achieve the same accuracy every time training is done. As stated in chapter \ref{chapter:datasets}, the seed for the random measurements the simulator does on a quantum circuit is not fixated, as this cannot be done on real hardware.\par 
The collected data shows that even a usable circuit that \emph{can} achieve perfect testing accuracy, as seen in figure \ref{fig:circuits_results_r_ap}, will often end up below that. When compared to the circuit used by Havlicek et al.\cite{havlicek_supervised_2019}, which was adapted to be trained with the same methods, it achieved near perfect \emph{or} perfect accuracy from the beginning of the training (on the artificial problem). This leads to the assumption, that some circuits might be easier to adapt to the problem space, and therefore, easier to learn, but cannot guarantee that a given circuit will always return the same results. This is also observable in the Heart Failure results when compared to the referred solution. Whilst cases exist, where it beats the comparison, most achieved scores are below the $0.933$ of said solution.\par 
Due to the limited availability of real hardware to perform long training sessions on, the training is done on the simulator with reduced amount of measurement shots (which are summed up to show the probabilities) of 1024. It can be argued that this is not enough to reach converging probabilities. When comparing to Havlicek et al. solution, it is clear that a circuit, which is more refined to the problem at hand, \emph{will} achieve better resulting accuracy, faster and with less variance, than general purpose circuits. It is not yet clear if such a circuit can be dynamically built for every problem space, or if they have to be carefully designed.\par
All figures in chapter \ref{chapter:results} display an overall increase in achieved accuracy from the performed normalization, which scaled down the feature space to the range $[-1,1]$. At the same time, it shows that there is a net loss of achieved accuracy when using a normalization with the range $[-\pi,\pi]$ on IRIS. An explanation is that any rotations that go beyond the point of $\ket{0}$ and $\ket{1}$ start to \emph{decrease} the probability of the given state again, even though initially, it was increasing it, as elaborated on in the chapter \ref{chapter:quantum_embedding}. When looking at the range of achieved accuracy for IRIS, the net gain for a normalization on $[-1,1]$ is in the range of $[0.020,0.028]$ and for the normalization on range $[-\pi,\pi]$ it was a net loss of $[-0.059,-0.054]$, when compared to the raw data scores. The same comparison on the heart failure data leads to a net gain of $[0.087, 0.229]$ for $[-1,1]$ and a net gain of $[0.070,0.131]$ for $[-\pi,\pi]$. \par
Comparing the raw features of the IRIS dataset to the ones of the Heart Failure dataset, the \emph{initial} range of feature values is larger for the Heart Failure dataset. Therefore, normalizing IRIS to $[-\pi,\pi]$ increases the total range of single features, which makes the feature space more \emph{sparse}, whilst in the Heart Failure dataset it decreases the range of single features, making it more dense. This explains why the classification of IRIS has a net loss and classifying Heart Failure a net gain when using a normalization of $[-\pi, \pi]$. \par
Collected data suggests that given an enough complex quantum circuit, it is possible to achieve excellent classification results for any given dataset. This can be seen in figures \ref{fig:circuits_results_non_r_iris} up to \ref{fig:circuits_results_r_ap}, where circuit $4$ managed to achieve good results across all datasets. This is supported by the findings of Sim et al.\cite{sim_expressibility_2019}, which show that combinations of gates can make use of a wider range of possible states, therefore making them more capable of trespassing the problem space.
Observing the results of the classification on real quantum hardware in figure \ref{fig:boxplot_real_hardware}, the measured accuracy is $~0.55$ \emph{less} than the one achieved in the simulator. There are several potential reasons for this. Initially, some variance is introduced from the setup through the limited number of shots. In addition, execution on real quantum hardware is always prone to noise which can falsify states, and therefore, lead to wrong measurements. This is an active problem that is still being tackled\cite{georgopoulos_modelling_2021,shaib_efficient_2021}, so future improvements in reliability of measurements are sure to increase classification scores.\par
A comparison of time complexity between our circuits and classical neural networks is difficult to create. Not only the lack of reliable references for this, but also flexibility of not only neural networks, but other machine learning algorithms too. These are commonly adapted to their classification problems, which makes a generic assumption of $O(x)$ unreliable. In addition, hardware with features that speed up neural networks\cite{baischer_learning_2021} add to the unreliability of such a comparison.

\clearpage
\section{Conclusion}
The results show that it is indeed possible to create a quantum circuit that has the ability to classify a wide range of datasets. They also show that even though they offer high adaptability to multiple problem spaces, the probabilistic measurements and that versatility can backfire whilst training circuits with the goal of classifying datasets. The implication is that there is a requirement for multiple evaluations under the same conditions before a circuit's capabilities can be assessed. Another takeaway from the conducted research is that limiting the expressiveness of the circuit with rotation gates that have fixed values attributed to them \emph{might} lead to less variance, as the optimizer cannot meddle with those values.