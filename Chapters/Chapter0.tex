\chapter{Computational Neuron} % Chapter title

\label{chapter:computational_neuron} % For referencing the chapter elsewhere, use \ref{chapter:computational_neuro} 

%----------------------------------------------------------------------------------------

% Define some commands to keep the formatting separated from the content 
\newcommand{\keyword}[1]{\textbf{#1}}
\newcommand{\tabhead}[1]{\textbf{#1}}
\newcommand{\code}[1]{\texttt{#1}}
\newcommand{\file}[1]{\texttt{\bfseries#1}}
\newcommand{\option}[1]{\texttt{\itshape#1}}

%----------------------------------------------------------------------------------------

\section{Classical Variant}
The classic computational neuron, often referred to as perceptron, is a multi-step process that amounts to a single result that can be used in further neurons, or directly read out as an output. In their simplest form, the neuron has $n$ input features that are multiplied with $n$ weights. In addition to those features, there is a weighted bias, where the feature itself has the fixed value $1$. These values are then summarized and evaluated through an activation function $f(x)$. Originally proposed by Rosenblatt F.\cite{rosenblatt_perceptron_1958}, the structure of neuron with $n = 4$ input features and one bias feature is shown in figure \ref{figure:computation_neuron}\footnote{There are many iterations of this design that range from a different handling of the bias to multiple output designs like LSTMs\cite{lstm}}. A more formal definition is presented in equation \ref{equation:computation_neuron}, where $j=0$ is the bias.

\begin{figure}[h!]
    \centering
    \begin{tikzpicture}[shorten >=1pt,node distance=1.5cm,on grid,auto]
    	\node (0) {$i_0$};
    	\node (1) [below=of 0] {$i_1$};
    	\node (2) [below=of 1] {$i_2$};
    	\node (3) [below=of 2] {$i_3$};
    	\node (4) [below=of 3] {$i_4$};
    	\node (5) [right=of 2] {$\Sigma$};
    	\node (6) [right=of 5] {$f(x)$};
    	\node (7) [right=of 6] {$y$};
    	\path[->]
    	(0) edge [bend left=45] node {$\omega_0$} (5)
    	(1) edge [bend left=25] node {$\omega_1$} (5)
    	(2) edge  node {$\omega_2$} (5)
    	(3) edge [bend right=25] node {$\omega_3$} (5)
    	(4) edge [bend right=45] node {$\omega_4$} (5)
    	(5) edge node {} (6)
    	(6) edge node {} (7);
    \end{tikzpicture}
    \caption{Computational neuron with 4 features and 1 bias}
    \label{figure:computation_neuron}
\end{figure}

\begin{equation}
    \centering
    y =\ f\left(\sum_{j = 0}^n i_j\omega_j\right)
    \label{equation:computation_neuron}
\end{equation}

\newpage

The activation function $f(x)$, also called transfer function, can be selected from a plethora of valid choices\cite{szandala_review_2021}, depending on the problem at hand. Dissecting the neuron leads to the two key components needed for the creation of an equivalent quantum neuron: A function for addition $U_{+}$ and one for multiplication $U_{\times}$ (The activation function can then be constructed from these). Whilst a single qubit can be in a coherent superposition of two different states\cite{nielsen_quantum_2010}, its measurements always amount to either 0 or 1, with their respective probabilities. Due to this the implementation uses binary data input. The implementation also neglects the usage of \emph{entanglement} and \emph{superpositions} as the arithmetic functions do not make usage of probabilistic calculations.

%------------------------------------------------------------------------------------------
\section{Binary Addition}\label{chapter:binary_addition}

Binary addition allows the combination of all components $i_j\omega_j$ into a single variable $x$ used in the summation. It is also a prerequisite to creating the binary multiplication circuit. The rules that define binary addition are relatively simple and are summarized in table \ref{table:binary_addition_rules} for the expression $x_A + x_B =\ y$\\
\begin{table}[!h]
\centering
\begin{tabular}{|c|c|c|c|}
\hline
$x_A$ & $x_B$ & $y$ \\ \hline
0    & 0    & 0  \\ \hline
0    & 1    & 1  \\ \hline
1    & 0    & 1  \\ \hline
1    & 1    & 0  \\ \hline
\end{tabular}
\caption{Binary addition of two $1$ bit wide words}
\label{table:binary_addition_rules}
\end{table}

With table \ref{table:binary_addition_rules} a matrix $U_{+}$ can be created, that follows the given rules that apply to the creation of quantum gates\cite{scherer_quantum_2019}. Simply speaking, the matrix has to fulfill $U_{+}U_{+}^\dagger = I$. This is also often referred to as \emph{bijective} when working with functions. It dictates that any single resulting state can be followed back to a single input state. To fulfill these prerequisites, we have to use 3 qubits to assert no loss of information. The first two qubits are input bits $x_A$ and $x_B$, respectively, whilst the third one is output bit $y$. In this case, the vector $\vec{S}$ stands for any of the $2^3 = 8$ possible \emph{measurement} states the 3 qubits can be in, and vector $\vec{O}$ to any possible output state, as shown in equation \ref{equation:basic_quantum_calculation}.\\

\begin{equation}
    U_{+}\vec{S} = \vec{O} \rightarrow U_{+} =\ OS^T
    \label{equation:basic_quantum_calculation}
\end{equation}

First we create the state matrix $S$ that represents all possible input states of our quantum circuit.

\begin{equation}
    \begin{split}
        \vec{S_0} =\ \ket{0} \otimes \ket{0} \otimes \ket{0} &=\ \begin{pmatrix} 1 \\ 0\end{pmatrix}\ \otimes \begin{pmatrix}1 \\ 0\end{pmatrix} \otimes \begin{pmatrix}1 \\ 0\end{pmatrix} =\ \begin{pmatrix}1 \\0 \\0 \\0 \\0 \\0 \\0 \\0 \end{pmatrix}\\
        \rightarrow S &= I^{8\times8}
    \end{split}
    \label{equation:state_matrix_is_identity_matrix}
\end{equation}

Using the same method and the information from table \ref{table:binary_addition_rules} we construct the output state matrix $O$ in equation \ref{equation:addition_output_matrix}.

\begin{equation}
    \begin{split}
        \text{1. Row:}\ x_{A} + x_{B} = y \rightarrow \ket{0} + \ket{0} = \ket{0}\\    
        \text{2. Row:}\ x_{A} + x_{B} = y \rightarrow \ket{0} + \ket{1} = \ket{1}\\    
        \text{3. Row:}\ x_{A} + x_{B} = y \rightarrow \ket{1} + \ket{0} = \ket{1}\\    
        \text{4. Row:}\ x_{A} + x_{B} = y \rightarrow \ket{1} + \ket{1} = \ket{0}\\    
    \end{split}
\end{equation}

\begin{equation}
\centering
    \begin{split}
        \vec{O}_0 &=\ \ket{0} \otimes \ket{0} \otimes \ket{0} =\ \begin{pmatrix}1 \\0 \\0 \\0 \\0 \\0 \\0 \\0 \end{pmatrix} \\
        \rightarrow O &=\ \begin{pmatrix}
        1 & 0 & 0 & 0 & 0 & 0 & 0 & 0\\
        0 & 0 & 0 & 0 & 0 & 1 & 0 & 0\\
        0 & 0 & 0 & 0 & 0 & 0 & 1 & 0\\
        0 & 0 & 0 & 1 & 0 & 0 & 0 & 0\\
        0 & 0 & 0 & 0 & 1 & 0 & 0 & 0\\
        0 & 1 & 0 & 0 & 0 & 0 & 0 & 0\\
        0 & 0 & 1 & 0 & 0 & 0 & 0 & 0\\
        0 & 0 & 0 & 0 & 0 & 0 & 0 &1
        \end{pmatrix}
    \end{split}
    \label{equation:addition_output_matrix}
\end{equation}

Note that the matrix $O$ in equation \ref{equation:addition_output_matrix} can be flipped, depending on what qubit is selected as output. The two matrices $O$ and $S$ can now be inserted into the equation $U_{+} = OS^T$ and solved for $U_{+}$, as per equation \ref{equation:addition_matrix_solved}

\begin{equation}
    \begin{split}
        U_{+} =\ OS^T &=\ \begin{pmatrix}
        1 & 0 & 0 & 0 & 0 & 0 & 0 & 0\\
        0 & 0 & 0 & 0 & 0 & 1 & 0 & 0\\
        0 & 0 & 0 & 0 & 0 & 0 & 1 & 0\\
        0 & 0 & 0 & 1 & 0 & 0 & 0 & 0\\
        0 & 0 & 0 & 0 & 1 & 0 & 0 & 0\\
        0 & 1 & 0 & 0 & 0 & 0 & 0 & 0\\
        0 & 0 & 1 & 0 & 0 & 0 & 0 & 0\\
        0 & 0 & 0 & 0 & 0 & 0 & 0 &1
        \end{pmatrix}\begin{pmatrix}
         1 & 0 & 0 & 0 & 0 & 0 & 0 & 0\\
         0 & 1 & 0 & 0 & 0 & 0 & 0 & 0\\ 
         0 & 0 & 1 & 0 & 0 & 0 & 0 & 0\\
         0 & 0 & 0 & 1 & 0 & 0 & 0 & 0\\
         0 & 0 & 0 & 0 & 1 & 0 & 0 & 0\\
         0 & 0 & 0 & 0 & 0 & 1 & 0 & 0\\
         0 & 0 & 0 & 0 & 0 & 0 & 1 & 0\\
         0 & 0 & 0 & 0 & 0 & 0 & 0 & 1\\
        \end{pmatrix} \\ 
        \rightarrow U_{+} &=\ \begin{pmatrix}
        1 & 0 & 0 & 0 & 0 & 0 & 0 & 0\\
        0 & 0 & 0 & 0 & 0 & 1 & 0 & 0\\
        0 & 0 & 0 & 0 & 0 & 0 & 1 & 0\\
        0 & 0 & 0 & 1 & 0 & 0 & 0 & 0\\
        0 & 0 & 0 & 0 & 1 & 0 & 0 & 0\\
        0 & 1 & 0 & 0 & 0 & 0 & 0 & 0\\
        0 & 0 & 1 & 0 & 0 & 0 & 0 & 0\\
        0 & 0 & 0 & 0 & 0 & 0 & 0 &1
        \end{pmatrix}
    \end{split}
    \label{equation:addition_matrix_solved}
\end{equation}
\newpage

Using IBMs \code{qiskit} the calculated matrix $U_{+}$ is turned into a quantum gate.

\begin{minted}{python}
from qiskit import QuantumCircuit
import qiskit.quantum_info as qi

n_qubits = 3

#Create the Gate
addition = qi.Operator([[1, 0, 0, 0, 0, 0, 0, 0],
                        [0, 0, 0, 0, 0, 1, 0, 0],
                        [0, 0, 0, 0, 0, 0, 1, 0],
                        [0, 0, 0, 1, 0, 0, 0, 0],
                        [0, 0, 0, 0, 1, 0, 0, 0],
                        [0, 1, 0, 0, 0, 0, 0, 0],
                        [0, 0, 1, 0, 0, 0, 0, 0],
                        [0, 0, 0, 0, 0, 0, 0, 1]])
\end{minted}

Binary addition that allows for bit words longer than one use a carry bit, which is then used as the input for the next addition. As the carry over is not always needed, we will use a separate quantum gate to calculate it. The rules for binary addition with carry over are listed in table \ref{table:binary_addition_carry_over}

\begin{table}[!h]
    \centering
    \begin{tabular}{|c|c|c|c|c|}
        \hline
        $carry_{in}$ & $x_A$ & $x_B$ & $y$ & $carry_{out}$ \\ \hline
        0             & 0     & 0     & 0   & 0              \\ \hline
        1             & 0     & 0     & 1   & 0              \\ \hline
        0             & 0     & 1     & 1   & 0              \\ \hline
        1             & 0     & 1     & 0   & 1              \\ \hline
        0             & 1     & 0     & 1   & 0              \\ \hline
        1             & 1     & 0     & 0   & 1              \\ \hline
        0             & 1     & 1     & 0   & 1              \\ \hline
        1             & 1     & 1     & 1   & 1              \\ \hline
        \end{tabular}
    \caption{Binary addition with carry over}
    \label{table:binary_addition_carry_over}
\end{table}

\newpage


As part $x_A + x_B = y$ is solved, only a solution for the $carry_{out}$ is created. It is important to note here that the $carry_{in}$ bit from table \ref{table:binary_addition_carry_over} is being reused as output bit $y$. Using the \emph{Toffoli Gate}, also called \code{CCNOT}\cite{qiskit_ccxgate_nodate}, we can use two controlling qubits to invert the state of a third one as demonstrated with figure \ref{fig:carry_over_circuit}.

\begin{figure}[!h]
    \centering
    \scalebox{1.0}{
        \Qcircuit @C=1.0em @R=0.8em @!R {
            \nghost{ {q}_{0} :  } & \lstick{ x_{A} :  } & \ctrl{1} & \qw & \ctrl{2} & \qw & \qw\\
            \nghost{ {q}_{1} :  } & \lstick{ x_{B} :  } & \ctrl{2} & \ctrl{1} & \qw & \qw & \qw\\
            \nghost{ {q}_{2} :  } & \lstick{ carry_{in}/y :      } & \qw & \ctrl{1} & \ctrl{1} & \qw & \qw\\
            \nghost{ {q}_{3} :  } & \lstick{ carry_{out} :  } & \targ & \targ & \targ & \qw & \qw\\ 
        }
    }
    \caption{Quantum circuit to calculate the carry over bit}
    \label{fig:carry_over_circuit}
\end{figure}

Combining these two solutions, we construct a quantum circuit to add any $n$ bit wide words together. Figure \ref{figure:four_bit_full_adder} demonstrates a circuit for the addition of two $4$ bit wide words\footnote{Note that we have replaced the X gates used previously to set the initial state with parameterized RX gates, where any variable that is $z = \pi$ leads to $\ket{1}$ and $z = 0$ to $\ket{0}$}.

\begin{figure}[!h]
    \centering
    \scalebox{0.5}{
    \Qcircuit @C=1.0em @R=0.2em @!R { \\
        \ghost{ {q}_{0} :  } & \lstick{ x_{A0} :  } & \gate{\mathrm{R_X}\,(\mathrm{a{\ensuremath{\theta}}})} \barrier[0em]{12} & \qw & \multigate{3}{\mathrm{carry\,over}}_<<<{0} & \multigate{2}{\mathrm{addition}}_<<<{0} & \qw & \qw & \qw & \qw & \qw & \qw & \qw & \qw & \qw\\ 
        \ghost{ {q}_{1} :  } & \lstick{ x_{B0} :  } & \gate{\mathrm{R_X}\,(\mathrm{b{\ensuremath{\theta}}})} & \qw & \ghost{\mathrm{carry\,over}}_<<<{1} & \ghost{\mathrm{addition}}_<<<{1} & \qw & \qw & \qw & \qw & \qw & \qw & \qw & \qw & \qw\\ 
        \ghost{ {q}_{2} :  } & \lstick{ y_{0} :  } & \qw & \qw & \ghost{\mathrm{carry\,over}}_<<<{2} & \ghost{\mathrm{addition}}_<<<{2} & \meter & \qw & \qw & \qw & \qw & \qw & \qw & \qw & \qw\\ 
        \ghost{ {q}_{3} :  } & \lstick{ carry_0 :  } & \qw & \qw & \ghost{\mathrm{carry\,over}}_<<<{3} & \multigate{3}{\mathrm{carry\,over}}_<<<{2} & \qw & \multigate{2}{\mathrm{addition}}_<<<{2} & \meter & \qw & \qw & \qw & \qw & \qw & \qw\\ 
        \ghost{ {q}_{4} :  } & \lstick{ x_{A1} :  } & \gate{\mathrm{R_X}\,(\mathrm{c{\ensuremath{\theta}}})} & \qw & \qw & \ghost{\mathrm{carry\,over}}_<<<{0} & \qw & \ghost{\mathrm{addition}}_<<<{0} & \qw & \qw & \qw & \qw & \qw & \qw & \qw\\ 
        \ghost{ {q}_{5} :  } & \lstick{ x_{B1} :  } & \gate{\mathrm{R_X}\,(\mathrm{d{\ensuremath{\theta}}})} & \qw & \qw & \ghost{\mathrm{carry\,over}}_<<<{1} & \qw & \ghost{\mathrm{addition}}_<<<{1} & \qw & \qw & \qw & \qw & \qw & \qw & \qw\\ 
        \ghost{ {q}_{6} :  } & \lstick{ carry_1 :  } & \qw & \qw & \qw & \ghost{\mathrm{carry\,over}}_<<<{3} & \qw & \multigate{3}{\mathrm{carry\,over}}_<<<{2} & \qw & \multigate{2}{\mathrm{addition}}_<<<{2} & \meter & \qw & \qw & \qw & \qw\\ 
        \ghost{ {q}_{7} :  } & \lstick{ x_{A2} :  } & \gate{\mathrm{R_X}\,(\mathrm{e{\ensuremath{\theta}}})} & \qw & \qw & \qw & \qw & \ghost{\mathrm{carry\,over}}_<<<{0} & \qw & \ghost{\mathrm{addition}}_<<<{0} & \qw & \qw & \qw & \qw & \qw\\ 
        \ghost{ {q}_{8} :  } & \lstick{ x_{B2} :  } & \gate{\mathrm{R_X}\,(\mathrm{f{\ensuremath{\theta}}})} & \qw & \qw & \qw & \qw & \ghost{\mathrm{carry\,over}}_<<<{1} & \qw & \ghost{\mathrm{addition}}_<<<{1} & \qw & \qw & \qw & \qw & \qw\\ 
        \ghost{ {q}_{9} :  } & \lstick{ carry_2 :  } & \qw & \qw & \qw & \qw & \qw & \ghost{\mathrm{carry\,over}}_<<<{3} & \qw & \multigate{3}{\mathrm{carry\,over}}_<<<{2} & \qw & \multigate{2}{\mathrm{addition}}_<<<{2} & \meter & \qw & \qw\\ 
        \ghost{ {q}_{10} :  } & \lstick{ x_{A3} :  } & \gate{\mathrm{R_X}\,(\mathrm{g{\ensuremath{\theta}}})} & \qw & \qw & \qw & \qw & \qw & \qw & \ghost{\mathrm{carry\,over}}_<<<{0} & \qw & \ghost{\mathrm{addition}}_<<<{0} & \qw & \qw & \qw\\ 
        \ghost{ {q}_{11} :  } & \lstick{ x_{B3} :  } & \gate{\mathrm{R_X}\,(\mathrm{h{\ensuremath{\theta}}})} & \qw & \qw & \qw & \qw & \qw & \qw & \ghost{\mathrm{carry\,over}}_<<<{1} & \qw & \ghost{\mathrm{addition}}_<<<{1} & \qw & \qw & \qw\\ 
        \ghost{ {q}_{12} :  } & \lstick{ carry_3 :  } & \qw & \qw & \qw & \qw & \qw & \qw & \qw & \ghost{\mathrm{carry\,over}}_<<<{3} & \qw & \qw & \qw & \qw & \qw\\ 
        \ghost{c:} & \lstick{c:} & \lstick{/_{_{4}}} \cw & \cw & \cw & \cw & \dstick{_{_{0}}} \cw \cwx[-11] & \cw & \dstick{_{_{1}}} \cw \cwx[-10] & \cw & \dstick{_{_{2}}} \cw \cwx[-7] & \cw & \dstick{_{_{3}}} \cw \cwx[-4] & \cw & \cw\\ 
        }
    }
    \caption{$4$ bit wide full adder\todo{add more explanation to image}}
    \label{figure:four_bit_full_adder}
\end{figure}

In figure \ref{figure:four_bit_full_adder} the $i$ in $x_{Ai}$ and $x_{Bi}$ stands for the $i$th bit in the word, starting with the LSB\footnote{Least significant bit}. Both the \emph{carry over} and \emph{addition} gates are the previously discussed ones. The numbering on their input side resembles the mapping of the incoming qubit to the \emph{internal qubit}. The figure also illustrates how the $carry_i$ qubit is reused as output qubit of the next addition step.

%----------------------------------------------------------------------------------------
\newpage

\section{Binary multiplication}
\label{chapter:binary_multiplication}

Using the same steps as in chapter \ref{chapter:binary_addition}, table \ref{table:binary_multiplication_rules} illustrates the binary multiplication rules.

\begin{table}[!h]
\centering
\begin{tabular}{|l|l|l|}
\hline
$x_A$ & $x_B$ & $y$ \\ \hline
0     & 0     & 0   \\ \hline
0     & 1     & 0   \\ \hline
1     & 0     & 0   \\ \hline
1     & 1     & 1   \\ \hline
\end{tabular}
\caption{Binary multiplication rules}
\label{table:binary_multiplication_rules}
\end{table}

The defined rules resemble only a subset of what makes binary multiplication work. The results of the multiplication get shifted one by one and then summarized. Figure \ref{figure:binary_multiplication_example} illustrates the procedure with a basic example.

\begin{figure}[!h]
\centering
\begin{tabular}{clllccllll}
$x_A$                   &  &   &   & 1                     & 0                     & 1 & 1 & 0 & 1 \\
$x_B$                   &  &   &   &                       &                       &   & 1 & 0 & 1 \\ \hline
                        &  &   &   & 1                     & 0                     & 1 & 1 & 0 & 1 \\
                        &  &   & 0 & 0                     & 0                     & 0 & 0 & 0 & X \\
$+$                     &  & 1 & 0 & 1                     & 1                     & 0 & 1 & X & X \\ \hline
\multicolumn{1}{l}{$y$} &  & 1 & 1 & \multicolumn{1}{l}{1} & \multicolumn{1}{l}{0} & 0 & 0 & 0 & 1
\end{tabular}
\caption{Binary multiplication example}
\label{figure:binary_multiplication_example}
\end{figure}

Combining quantum rules and our multiplication ruleset creates a problem, as illustrated in table \ref{table:three_qubit_state_multiplications}

\begin{table}[!h]
\centering
\begin{tabular}{|l|l|}
\hline
$\ket{\psi_{in}}$ & $\ket{\psi_{out}}$ \\ \hline
000               & 000                \\ \hline
001               & 000                \\ \hline
010               & 010                \\ \hline
011               & 010                \\ \hline
100               & 100                \\ \hline
101               & 100                \\ \hline
110               & 111                \\ \hline
111               & 111                \\ \hline
\end{tabular}
\caption{3 qubit state multiplications}
\label{table:three_qubit_state_multiplications}
\end{table}

Table \ref{table:three_qubit_state_multiplications} shows multiple states in $\ket{\psi_{out}}$ that are duplicates, which, as discussed in chapter \ref{chapter:binary_addition}, is not allowed. With this information at hand we have to \emph{cheat} the system by assuming that any implementation will make sure the $y_{in}$ is always $\ket{0}$. Table \ref{table:three_qubit_state_multiplications_corrected} is corrected for the new behaviour which fulfills all requirements.

\begin{table}[!h]
\centering
\begin{tabular}{|l|l|}
\hline
$\ket{\psi_{in}}$ & $\ket{\psi_{out}}$ \\ \hline
000               & 000                \\ \hline
001               & 001                \\ \hline
010               & 010                \\ \hline
011               & 011                \\ \hline
100               & 100                \\ \hline
101               & 101                \\ \hline
110               & 111                \\ \hline
111               & 110                \\ \hline
\end{tabular}
\caption{3 qubit state multiplications corrected}
\label{table:three_qubit_state_multiplications_corrected}
\end{table}

\newpage

The solution is not perfect, but binary multiplication does not use $y_{in} = 1$ in its rule set and therefor, any implementation where that behavior does happen, is either on \emph{purpose} or due to an implementation error. As a security measure, later implementations could simply reset $y_{in}$ before any calculations are done. Equation \ref{equation:state_matrix_is_identity_matrix} demonstrates that the state matrix $S$ equals the identitiy matrix $I$ of the same dimensionality (as we have 3 qubits, we get $2^3 = 8$). The output matrix $O$ is constructed using the defined states in table \ref{table:three_qubit_state_multiplications_corrected} and then used to calculate $U_{\times}$ as shown in equation \ref{equation:multiplication_matrix_solved}.

\begin{equation}
    \begin{split}
        U_{\times} =\ OS^T &=\ \begin{pmatrix}
         1 & 0 & 0 & 0 & 0 & 0 & 0 & 0\\
         0 & 1 & 0 & 0 & 0 & 0 & 0 & 0\\
         0 & 0 & 1 & 0 & 0 & 0 & 0 & 0\\
         0 & 0 & 0 & 0 & 0 & 0 & 0 & 1\\
         0 & 0 & 0 & 0 & 1 & 0 & 0 & 0\\
         0 & 0 & 0 & 0 & 0 & 1 & 0 & 0\\
         0 & 0 & 0 & 0 & 0 & 0 & 1 & 0\\
         0 & 0 & 0 & 1 & 0 & 0 & 0 & 0\\
         \end{pmatrix}\begin{pmatrix}
         1 & 0 & 0 & 0 & 0 & 0 & 0 & 0\\
         0 & 1 & 0 & 0 & 0 & 0 & 0 & 0\\ 
         0 & 0 & 1 & 0 & 0 & 0 & 0 & 0\\
         0 & 0 & 0 & 1 & 0 & 0 & 0 & 0\\
         0 & 0 & 0 & 0 & 1 & 0 & 0 & 0\\
         0 & 0 & 0 & 0 & 0 & 1 & 0 & 0\\
         0 & 0 & 0 & 0 & 0 & 0 & 1 & 0\\
         0 & 0 & 0 & 0 & 0 & 0 & 0 & 1\\
        \end{pmatrix} \\ 
        \rightarrow U_{\times} &=\ \begin{pmatrix}
         1 & 0 & 0 & 0 & 0 & 0 & 0 & 0\\
         0 & 1 & 0 & 0 & 0 & 0 & 0 & 0\\
         0 & 0 & 1 & 0 & 0 & 0 & 0 & 0\\
         0 & 0 & 0 & 0 & 0 & 0 & 0 & 1\\
         0 & 0 & 0 & 0 & 1 & 0 & 0 & 0\\
         0 & 0 & 0 & 0 & 0 & 1 & 0 & 0\\
         0 & 0 & 0 & 0 & 0 & 0 & 1 & 0\\
         0 & 0 & 0 & 1 & 0 & 0 & 0 & 0\\
         \end{pmatrix}
    \end{split}
    \label{equation:multiplication_matrix_solved}
\end{equation}

\newpage
The calculated matrix $U_{\times}$ is then applied in \code{qiskit}, which creates the corresponding gate.

\begin{minted}{python}
from qiskit import QuantumCircuit
import qiskit.quantum_info as qi

n_qubits = 3

#Create the Gate
multiplication = qi.Operator([[1, 0, 0, 0, 0, 0, 0, 0],
                              [0, 1, 0, 0, 0, 0, 0, 0],
                              [0, 0, 1, 0, 0, 0, 0, 0],
                              [0, 0, 0, 0, 0, 0, 0, 1],
                              [0, 0, 0, 0, 1, 0, 0, 0],
                              [0, 0, 0, 0, 0, 1, 0, 0],
                              [0, 0, 0, 0, 0, 0, 1, 0],
                              [0, 0, 0, 1, 0, 0, 0, 0]])
\end{minted}

With $U_{\times}$ and $U_{\add}$ calculated and the \emph{carry over} gate defined, a full binary multiplier can be constructed, as shown in figure \ref{figure:two_bit_wide_multiplier}.

\begin{figure}[!h]
    \centering
        \scalebox{0.41}{
        \Qcircuit @C=1.0em @R=0.2em @!R { \\
        \ghost{ x_{A0} :  } & \lstick{ x_{A0} :  } & \gate{\mathrm{R_X}\,(\mathrm{a{\ensuremath{\theta}}})} \barrier[0em]{9} & \qw & \multigate{4}{\mathrm{multiplication}}_<<<{0} & \multigate{5}{\mathrm{multiplication}}_<<<{0} & \qw & \qw & \qw & \qw & \qw & \qw & \qw & \qw & \qw & \qw & \qw & \qw & \qw\\ 
        \ghost{ x_{A1} :  } & \lstick{ x_{A1} :  } & \gate{\mathrm{R_X}\,(\mathrm{b{\ensuremath{\theta}}})} & \qw & \ghost{\mathrm{multiplication}} & \ghost{\mathrm{multiplication}} & \multigate{5}{\mathrm{multiplication}}_<<<{0} & \qw & \multigate{6}{\mathrm{multiplication}}_<<<{0} & \qw & \qw & \qw & \qw & \qw & \qw & \qw & \qw & \qw & \qw\\ 
        \ghost{ x_{B0} :  } & \lstick{ x_{B0} :  } & \gate{\mathrm{R_X}\,(\mathrm{c{\ensuremath{\theta}}})} & \qw & \ghost{\mathrm{multiplication}}_<<<{1} & \ghost{\mathrm{multiplication}} & \ghost{\mathrm{multiplication}}_<<<{1} & \qw & \ghost{\mathrm{multiplication}} & \qw & \qw & \qw & \qw & \qw & \qw & \qw & \qw & \qw & \qw\\ 
        \ghost{ x_{B1} :  } & \lstick{ x_{B1} :  } & \gate{\mathrm{R_X}\,(\mathrm{d{\ensuremath{\theta}}})} & \qw & \ghost{\mathrm{multiplication}} & \ghost{\mathrm{multiplication}}_<<<{1} & \ghost{\mathrm{multiplication}} & \qw & \ghost{\mathrm{multiplication}}_<<<{1} & \qw & \qw & \qw & \qw & \qw & \qw & \qw & \qw & \qw & \qw\\ 
        \ghost{ y_0 :  } & \lstick{ y_0 :  } & \qw & \qw & \ghost{\mathrm{multiplication}}_<<<{2} & \ghost{\mathrm{multiplication}} & \ghost{\mathrm{multiplication}} & \meter & \ghost{\mathrm{multiplication}} & \gate{\mathrm{\left|0\right\rangle}} & \multigate{4}{\mathrm{carry\,over}}_<<<{2} & \multigate{2}{\mathrm{addition}}_<<<{2} & \meter & \gate{\mathrm{\left|0\right\rangle}} & \multigate{5}{\mathrm{carry\,over}}_<<<{2} & \multigate{4}{\mathrm{addition}}_<<<{2} & \meter & \qw & \qw\\ 
        \ghost{ {q}_{5} :  } & \lstick{ {q}_{5} :  } & \qw & \qw & \qw & \ghost{\mathrm{multiplication}}_<<<{2} & \ghost{\mathrm{multiplication}} & \qw & \ghost{\mathrm{multiplication}} & \qw & \ghost{\mathrm{carry\,over}}_<<<{0} & \ghost{\mathrm{addition}}_<<<{0} & \qw & \qw & \ghost{\mathrm{carry\,over}} & \ghost{\mathrm{addition}} & \qw & \qw & \qw\\ 
        \ghost{ {q}_{6} :  } & \lstick{ {q}_{6} :  } & \qw & \qw & \qw & \qw & \ghost{\mathrm{multiplication}}_<<<{2} & \qw & \ghost{\mathrm{multiplication}} & \qw & \ghost{\mathrm{carry\,over}}_<<<{1} & \ghost{\mathrm{addition}}_<<<{1} & \qw & \qw & \ghost{\mathrm{carry\,over}} & \ghost{\mathrm{addition}} & \qw & \qw & \qw\\ 
        \ghost{ {q}_{7} :  } & \lstick{ {q}_{7} :  } & \qw & \qw & \qw & \qw & \qw & \qw & \ghost{\mathrm{multiplication}}_<<<{2} & \qw & \ghost{\mathrm{carry\,over}} & \qw & \qw & \qw & \ghost{\mathrm{carry\,over}}_<<<{0} & \ghost{\mathrm{addition}}_<<<{0} & \qw & \qw & \qw\\ 
        \ghost{ {q}_{8} :  } & \lstick{ {q}_{8} :  } & \qw & \qw & \qw & \qw & \qw & \qw & \qw & \qw & \ghost{\mathrm{carry\,over}}_<<<{3} & \qw & \qw & \qw & \ghost{\mathrm{carry\,over}}_<<<{1} & \ghost{\mathrm{addition}}_<<<{1} & \qw & \qw & \qw\\ 
        \ghost{ y_1 :  } & \lstick{ y_1 :  } & \qw & \qw & \qw & \qw & \qw & \qw & \qw & \qw & \qw & \qw & \qw & \qw & \ghost{\mathrm{carry\,over}}_<<<{3} & \meter & \qw & \qw & \qw\\ 
        \ghost{c:} & \lstick{c:} & \lstick{/_{_{4}}} \cw & \cw & \cw & \cw & \cw & \dstick{_{_{0}}} \cw \cwx[-6] & \cw & \cw & \cw & \cw & \dstick{_{_{1}}} \cw \cwx[-6] & \cw & \cw & \dstick{_{_{3}}} \cw \cwx[-1] & \dstick{_{_{2}}} \cw \cwx[-6] & \cw & \cw\\ 
        \\ }}
    \caption{$2$ bit wide multiplier\todo{add more explanation to image}}
    \label{figure:two_bit_wide_multiplier}
\end{figure}

\newpage 

Figure \ref{figure:two_bit_wide_multiplier} shows a circuit that can multiply two $2$ bit wide words. The presented form is optimized, as in, the qubit that has been measured is reset and reused for the next calculation step. In comparison to an non-optimized solution, this reduces the amount of required qubits, in this example, from $12$ to $10$. Figure \ref{figure:two_bit_wide_multiplier} requires quantum hardware with \emph{at least} 10 qubits to run. Equations \ref{equation:1_bit_multiplier} and \ref{equation:2_bit_multiplier} demonstrate how the required amount of qubits can be calculated. $n_{bits} = 1$ is a special case, as the multiplications result is always a $1$ bit wide. With $n_{bits} > 1$ the results length $l$ is in the range $n_{bits} \leq l \leq 2n_{bits}$

\begin{equation}\label{equation:1_bit_multiplier}
\centering
    \begin{split}
        n_{bits} &= 1 \\
        n_{measure} &= 1\\
        n_{qubits} &= 2n_{bits} + n_{measure} = 3
    \end{split}
\end{equation}

$n_{bits} > 1$ adds a special rules that can be observed in figure \ref{figure:two_bit_wide_multiplier}. The intermediate results of every multiplication, except the first, must be preserved. Additional qubits are then needed to act as $carry$ for the binary addition. For this case, $n_{carry}$ equals $1$ as we can reuse the last measurement qubit. This was omitted in equation \ref{equation:1_bit_multiplier} as there was no need for binary addition. Note that $n_{measure}$ is always $2$ as we can reuse the first measurement, but need a second one for the last measurment.

\begin{equation}\label{equation:2_bit_multiplier}
\centering
    \begin{split}
        n_{bits} &= 2 \\
        n_{U_{\times}} &= 2n_{bits} -1 \\
        n_{measure} &= 2\\
        n_{carry} &= 1\\
        n_{qubits} &= 2n_{bits} + n_{U_{\times}} + n_{carry} + n_{measure} = 10
    \end{split}
\end{equation}

Any larger sizes could not be create, and therefor, evaluated. The required amount of qubits needed to create and simulate a circuit for a $3$ bit wide multiplier disallowed any experiments, as the \code{qiskit} simulator could not handle it. It is expected that the amount of qubits increases \emph{at least} linearly. The amount of binary additions required in the classical solution for $n_{bits}$ are listed in \ref{table:amount_addition_binary_multiplication}.

\begin{table}[!h]
    \centering
    \begin{tabular}{|c|c|}
    \hline
    $n_{bits}$ & amount of additions \\ \hline
    1          & 0                   \\ \hline
    2          & 1                   \\ \hline
    3          & 4                   \\ \hline
    4          & 9                   \\ \hline
    \end{tabular}
    \caption{Binary addition steps for $n_{bits}$ in a classical solution}
    \label{table:amount_addition_binary_multiplication}
\end{table}


\newpage

%-------------------------------------------

\section{Conclusion}

The solutions presented in this chapter show that replication of a computational neuron is possible. The amount of required qubits for the implementation of basic circuits seems a deterrent when further pursuing this solution. Current quantum solutions for classical problems like \emph{Multiple Query Optimization} as demonstrated by Fankhauser et al.\cite{fankhauser_multiple_2021} can be solved without the need to replicate classical methods with the usage of superpositions and entanglement, which are completely absent from the solutions presented in chapters \ref{chapter:binary_addition} and \ref{chapter:binary_multiplication}. 

This does not mean that the proposed circuits serve no purpose. These could, in the future, be used to allow basic arithmetic on quantum-only devices. The reversed state of the $n$-bit multiplier shown in figure \ref{figure:two_bit_wide_multiplier} could be adapted, using superpositions and entanglement, to calculate two factors used to calculate the given value.
\todo{Quantum vorteil?}